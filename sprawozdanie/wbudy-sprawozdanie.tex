\documentclass{classrep}
\usepackage[utf8]{inputenc}
\frenchspacing

\usepackage{graphicx}
\usepackage{subfig}
\usepackage{float}
\usepackage[usenames,dvipsnames]{color}
\usepackage[hidelinks]{hyperref}

\usepackage{adjustbox}

\usepackage{amsmath, amssymb, mathtools}

\usepackage{hyperref}

\usepackage{fancyhdr, lastpage}
\pagestyle{fancyplain}
\fancyhf{}
\renewcommand{\headrulewidth}{0pt}
\cfoot{\thepage\ / \pageref*{LastPage}}


\studycycle{Informatyka, studia dzienne, I st.}
\coursesemester{IV}

\coursename{Systemy wbudowane}
\courseyear{2018/2019}

\courseteacher{dr inż. Michał Morawski}
\coursegroup{środa, 12:15}

\author{%
	\textbf{Grupa D07}\\\\
	\studentinfo[216879@edu.p.lodz.pl]{Przemysław Rudowicz}{216879} - lider\\
	\studentinfo[216782@edu.p.lodz.pl]{Konrad Jaworski}{216782}\\
	\studentinfo[216866@edu.p.lodz.pl]{Jakub Plich}{216866}%
}

\title{Dokumentacja projektu gry Snake\\
		LPC1768/9}
\begin{document}
	\thispagestyle{fancyplain}
	\maketitle
	

	
	\newpage
	\tableofcontents
	\newpage
	
	\section{Podział obowiązków}
	{
	\subsection{Wykorzystane funkcjonalności}
		\begin{center}
			\begin{tabular}{|c|c|}
				\hline 
				\textbf{Funkcjonalność} & \textbf{Osoba za nią odpowiedzialna} \\ 
				\hline 
				GPIO (joystick) & Konrad Jaworski \\ 
				\hline 
				Akcelerometr & Konrad Jaworski \\ 
				\hline 
				Głośnik & Konrad Jaworski \\ 
				\hline 
				Timer & Przemysław Rudowicz \\ 
				\hline 
				OLED & Przemysław Rudowicz \\ 
				\hline 
				SSP/SPI & Przemysław Rudowicz \\ 
				\hline 
				Czujnik światła & Jakub Plich \\ 
				\hline 
				pca9532 & Jakub Plich \\ 
				\hline 
				I\textsuperscript{2}C & Jakub Plich \\ 
				\hline 
				Rotacyjny przełącznik kwadraturowy & Jakub Plich\\
				\hline
			\end{tabular} 	
		\end{center}
	}


	\subsection{Podział obowiązków}
\begin{center}
			\begin{tabular}{|c|c|}
			\hline 
			\textbf{Imię i nazwisko} & \textbf{Procentowy udział w pracy} \\ 
			\hline 
			Konrad Jaworski & 33\% \\ 
			\hline 
			Przemysław Rudowicz & 34\% \\ 
			\hline 
			Jakub Plich & 33\% \\ 
			\hline 
		\end{tabular} 
\end{center}
\newpage

	\section{Opis działania programu}
	
		\subsection{Instrukcja użytkownika}
		
		\subsection{Opis algorytmu}
		
	\section{Funkcjonalności}
		\subsection{GPIO}
		GPIO (oznacza general-purpose input/output) - interfejs wejścia/wyjścia ogólnego przeznaczenia. Należy ustawić kierunki wejścia/wyjścia pinów GPIO (0 - gdy chcemy skonfigurować pin jako wejście, lub 1 - jako wyjście).
		
		\subsubsection{Głośnik}
		Głośnik jest obsługiwany przy pomocy pinów GPIO.  Jako, że głośnik nie będzie wysyłał danych, piny ustawiamy na wyjście. W tym celu ustawiamy wartość 1 w rejestrach FIODIR0 i FIODIR2 w miejscach odpowiadających pinom głośnika (każdy bit rejestru odpowiada jednemu pinowi GPIO, każdy port GPIO ma swój rejestr FIODIR). A więc 1 należy ustawić na 28, 27, 26 bicie FIODIR0 i 13 bicie FIODIR2.
		
		
		Wzmacniacz analogowy LM4811, który znajduje się na płytce LPCXpresso Base Board potrzebuje następujących pinów z mikrokontrolera:
		\begin{itemize}
			\item CLK
			\item UP/DN
			\item SHUTDN
			\item VIN1/VIN2
		\end{itemize}
	
		\medskip
		Ze specyfikacji LM8411 \cite{LM4811} dowiadujemy się, że piny CLK (CLOCK) oraz UP/DN
		są odpowiedzialne za sterowanie głośnością brzęczyka.
		
		
		Pin SHUTDN aktywuje funkcję zmniejszającą pobór prądu przez wzmacniacz (Nie korzystamy z tej funkcji).
		
		
		Piny VIN1/VIN2 odpowiadają za generację sygnału wprawiającego
		membranę brzęczyka w drgania (generowanie dźwięków).\\
		
		
		\begin{center}
			Sposób połączenie pinów wzmacniacza analogowego do pinów GPIO:\smallskip

			\begin{tabular}{|c|c|}
				\hline 
				\textbf{Piny LM4811} & \textbf{Piny GPIO} \\ 
				\hline 
				CLK & P0.27 \\ 
				\hline 
				UP/DN & P0.28 \\ 
				\hline 
				SHUTDN & P2.13 \\ 
				\hline 
				VIN1/VIN2 & P0.26 \\ 
				\hline 
			\end{tabular} 
		\end{center}
	
		Podczas inicjalizacji głośnika czyszczone są wartość na pinach P0.27, P0.28, P2.13 (ustawiamy 1 w rejestrach FIOCLR dla portu 0 i 2 w miejscach odpowiadających wymienionym pinom).
		
		
		Generowanie dźwięku przez brzęczyka odbywa się poprzez podawaniu zmiennego napięcia na pin P0.26 tak aby
		wprowadzić membranę brzęczyka w drgania. Pozwala to na generowanie prostych nut. 
		
		
		Aby zagrać nutę 'C', należy wprowadzić membranę brzęczka w drgania o częstotliwości $f=262Hz$. A więc okres drań $T=\frac{1}{f} = 3816 \mu s$. Stąd na pinie P0.26 należy ustawić stan wysoki przez czas równy $\frac{T}{2} = 1908\mu s$ oraz stan niski analogicznie przez  $\frac{T}{2}$. Cykl należy powtarzać w zależności od tego jak długo chcemy odtwarzać dźwięk. Do ustawiania stanów wysokich i niskich na pinach GPIO używamy rejestru FIOSET i FIOCLR. Za
		generowanie dźwięku odpowiada pin P0.26. Analogicznie postępujemy w przypadku innych nut.
		
		W celu ustawienia stanu wysokiego na pinie P0.26 należy ustawić 1 na 26 bicie rejestru FIOSET (ustawianie zera na tym rejestrze nie ustawia stanu niskiego). Aby odwołać stan wysoki należy wpisać 1 na 26 bicie rejestru FIOCLR.
		\subsubsection{Joystick}
		Joystick również jest obsługiwany przy pomocy pinów GPIO. Natomiast w przeciwieństwie do głośnika, joystick wysyła dane do mikrokontrolera, a więc podczas jego inicjalizacji ustawiamy wszystkie piny na wejście.
		
		W tym celu ustawiamy wartość '0' na 15, 16 i 17 bicie rejestru FIODIR0 oraz na 3 i 4 bicie rejestru FIODIR2. 
		
\begin{center}
			\begin{tabular}{|c|c|c|}
			\hline 
			\textbf{Pozycja joysticka} & \textbf{wartość} & \textbf{Piny GPIO} \\ 
			\hline 
			JOYSTICK\_CENTER & 0x01 & P0.17 \\ 
			\hline 
			JOYSTICK\_UP & 0x02 & P2.3 \\ 
			\hline 
			JOYSTICK\_DOWN & 0x04 & P0.15 \\ 
			\hline 
			JOYSTICK\_LEFT & 0x08 & P2.4 \\ 
			\hline 
			JOYSTICK\_RIGHT & 0x10 & P0.16 \\ 
			\hline 
		\end{tabular} 
\end{center}
		
		Stany podłączonych pinów odpowiadają stanom wciśnięcia joysticka (odpowiednio tak jak w tabeli powyżej).
		W celu odczytania stanu joysticka sprawdzane są wartości na kolejnych pinach (odpowiednio tak jak w tabeli powyżej) i jeżeli jego wartość to '0', zmienna przechowująca stan joysticka przyjmuje wartość koniunkcji bitowej tego stanu i odpowiadającej wartości (patrz tabela powyżej) przypisanej do pozycji joysticka. 
		\subsection{Akcelerometr}
	Użyty przez nas akcelerometr MMA7455L \cite{MMA7455L} jest inicjalizowany (za pomocą I\textsuperscript{2}C (adres urządzenia - 0x1D)) przez zdefiniowanie trybu pomiaru i czułości. Dokonujemy tego poprzez wpisanie do rejestru MCTL (Mode Control o adresie 0x16) wartości 1 na bicie zerowym (odpowiada za ustawienie czułości na 2g) oraz wartości 1 na bicie drugim (odpowiada za ustawienie trybu pomiarowego (Measurement Mode)).
	
\begin{center}
		\resizebox{\textwidth}{!}{%
	\begin{tabular}{|c|c|c|c|c|c|}
		\hline 
		Adres & Nazwa rejestru & Bit 3 (GLVL[1]) & Bit 2 (GLVL[0]) & Bit 1 (MOD[1]) & Bit 0 (MOD[0]) \\ 
		\hline 
		0x16 & MCTL & 0 & 1 & 0 & 1 \\ 
		\hline 
	\end{tabular} 
	}
\end{center}

		Aby odczytać wartości wskazań akcelerometru najpierw zostaje sprawdzona wartość z rejestru przechowujące informację o statusie akcelerometru (rejestr o adresie 0x09). Na bicie zerowym tego rejestru (DRDY) przechowywana jest informacja, czy dane są gotowe do odczytania.
\begin{center}
			\begin{tabular}{|c|c|}
			\hline 
			\vtop{\hbox{\strut \textbf{Wartość na bicie DRDY}}\hbox{\strut \textbf{rejestru 0x09}}} & \textbf{Znaczenie} \\ 
			\hline 
			1 & dane są gotowe do odczytania \\ 
			\hline 
			0 & dane nie są gotowe do odczytania \\ 
			\hline 
		\end{tabular} 
\end{center}
		
		Jeżeli dane są gotowe do odczytania, z rejestru XOUT8 (o adresie 0x06) pobierane jest 8 bitów, które opisują składową X wektora przyspieszenia. Następnie, kolejno odczytywane są wartości z rejestrów YOUT8 (0x07) oraz ZOUT8 (0x08), które opisują odpowiednio składowe Y i Z. 
		
\begin{center}
			\begin{tabular}{|c|c|}
			\hline 
			\textbf{Adres rejestru} & \textbf{Nazwa rejestru} \\ 
			\hline 
			0x06 & XOUT8 \\ 
			\hline 
			0x07 & YOUT8 \\ 
			\hline 
			0x08 &  ZOUT8\\ 
			\hline 
		\end{tabular} 
\end{center}
			
		\subsection{Timer}
		
		\subsection{OLED}
		
		\subsection{SSP/SPI}
		
		\subsection{Czujnik światła}
Czujnik światła jest urządzeniem peryferyjnym przyłączonym do płytki magistralą I\textsuperscript2C. Program wkorzystuje odczytane natężenie światła do odwrócenia kolorów na 			wyświetlaczu w momencie w którym odczytana wartość natężenia światła będzie mniejsza niż 25 luksów. Uruchomienie czyjnika światła polega na przesłaniu przez	
I2C pod adres (0x44) kolejno wartości (0x00) oraz (1<<7) . To powoduje ustawienie wartości 1 na 7 bicie rejestru Command Register(0x00) i w konsekwencji 	uruchomienie przetwornika analogowo-cyfrowego w czujniku. Zakres odczytu czujnika jest domyślny i wynosi od 0 do 1000 luksów. Odczyt wartości pomiaru czujnika wymaga 				odczytania zawartości dwóch rejestrów:  LSB-Sensor - zawiera dolny bajt ostatniego odczytu sensora(adres 0x04), MSB-Sensor - zawiera górny bajt ostatniego odczytu 					sensora(adres 0x05). Wynik wyrażony w luksach jest obliczany z następującego wzoru: E = 973 * odczytana-wartość /  (1<<16)
\subsection{PCA9532}
Expander PCA9532 wyposażony w 16 diód LED jest wkorzystywany do reprezentacji wyniku w danym momencie gry. Zapalone diody przedstawiają wynik w postaci binarnej.
W celu zapalenia odpowiednich diód należy ustalić 16 bitową maskie w której wartości 1 oznaczają zapaloną diodę. Następnie przez I2C pod adres rejestru kontrolnego(0x60)  				zostaje wysłany bajt kontrolny. Trzeci  bit bajtu kontorlnego onacza flagę inkrementacji która zwiększa o 1 adres podany w pozostałych 4 bitach po każdym przesłanym bajcie. 				Potem przesłane zostają 4 bajty które zostają kolejno wpisane do 4  8-bitowych rejestrów(LS0,LS1,LS2,LS3) w których każde 2 bity odpowiadają jednej diodzie. Program korzysta 				jedynie ze stanów OFF(00) oraz ON(01).
\subsection{I\textsuperscript{2}C}
I2C (Inter-Integrated Circuit) to szeregowa, dwukierunkowa magistrala do przesyłania danych. Program wykorzystuje ją do komunikacji z PCA9532, czujnikiem światła oraz 					akcelerometrem.
Początek inicjalizacji magistrali zaczyna się od konfiguracji pinów które będą pełniły fukcjię lini SCL(linia zegara) i SDA(linia danych). Dla interfejsu I2C2 są one ustawione 				odpowiednio na pinach P0[11] oraz P0[10]. W tym celu ustawiamy w rejestrze PINSEL0 wartosci 1 i 0 kolejno dla bitów 21 i 20(P0.10 SDA2) oraz wartość 1 i 0 dla bitów 23 i 22(P0.11 		SCL). Następnie następuje włączenie zasilania dla I2C poprzez ustawienie wartości 1 na 26 bicie rejestru PCONP(Power Control for Peripherals Register). Ustawiamy dzielnik zegara 			PCLK na 2 ustawiając w rejestrze PCLKSEL 20 i 21 bit na wartości kolejno 0 i 1. Następnie należy ustawić wartość rejestrów I2SCLH i I2SCLL na żądaną ilość cykli zegara PCLK. Obie			 	te wartości są sobie równe.(...)
 
Na koniec występuje ustawienie na wartość 1 bitów 2(flaga AA), 5(flaga STA) i 6(flaga I2EN) rejestru CONCLR.
Na koniec należy w rejestrze I2CONSET ustawiamy wartość 1 na 6 bicie aby włączyć interfejs I2C2. 
\subsection{Rotacyjny przełącznik kwadraturowy}
Rotacyjny przełącznik kwadraturowy jest wykorzystywany w naszym programie do regulacji prędkości poruszania się "węża". Obrót przełącznika w prawo zmniejsza prędkość, a obór w lewo zwiększa. Inicjalizacja ogranicza się do ustawienia na porcie 0 GPIO wartości 0 na 24 i 25 bicie rejestru FIODIR(zerowy i pierwszy bit rejestru FIO0DIR3). Wartość 0 oznacza że pin jest ustawiony na wejście. Program odczytuje inforamcję o 4 stanach(są ustalone na podstawie położenia przełącznika). W zależności od kolejności występowania stanów ustalany jest kierunek ruchu przełącznika.
	\section{Analiza FMEA}
	

	\resizebox{\textwidth}{!}{%
	\hspace*{-85pt}\begin{tabular}{|c|c|c|c|c|c|}
		\hline 
		Ryzyko & Prawdopodobieństwo & Znaczenie & (Samo)Wykrywalność & Iloczyn & Reakcja \\ 
		\hline 
		Uszkodzenie joysticka &  &  &  &  &  \\ 
		\hline 
	\end{tabular} 
}


\begin{thebibliography}{abcd}
	\bibitem{MMA7455L}{
		\textit{$\pm$2g/$\pm$24g/$\pm$28g Three Axis Low-g
			Digital Output Accelerometer},
		{Rev 8, 07/2009},
		\textbf{Freescale Semiconductor}
	}	

	\bibitem{LM4811}{
	\textit{LM4811
		Dual 105mW Headphone Amplifier with Digital Volume
		Control and Shutdown Mode Datasheet},
	{December 2002},
	\textbf{National Semiconductor}
}		
\end{thebibliography}

\end{document}
