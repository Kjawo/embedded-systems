\documentclass{classrep}
\usepackage[utf8]{inputenc}
\frenchspacing

\usepackage{graphicx}
\usepackage{subfig}
\usepackage{float}
\usepackage[usenames,dvipsnames]{color}
\usepackage[hidelinks]{hyperref}

\usepackage{adjustbox}

\usepackage{amsmath, amssymb, mathtools}

\usepackage{hyperref}

\usepackage{fancyhdr, lastpage}
\pagestyle{fancyplain}
\fancyhf{}
\renewcommand{\headrulewidth}{0pt}
\cfoot{\thepage\ / \pageref*{LastPage}}


\studycycle{Informatyka, studia dzienne, I st.}
\coursesemester{IV}

\coursename{Systemy wbudowane}
\courseyear{2018/2019}

\courseteacher{dr inż. Michał Morawski}
\coursegroup{środa, 12:15}

\author{%
	\textbf{Grupa D07}\\\\
	\studentinfo[216879@edu.p.lodz.pl]{Przemysław Rudowicz}{216879} - lider\\
	\studentinfo[216782@edu.p.lodz.pl]{Konrad Jaworski}{216782}\\
	\studentinfo[216866@edu.p.lodz.pl]{Jakub Plich}{216866}%
}

\title{Dokumentacja projektu gry Snake\\
		LPC1768/9}
\begin{document}
	\thispagestyle{fancyplain}
	\maketitle
	

	
	\newpage
	\tableofcontents
	\newpage
	
	\section{Podział obowiązków}
	{
	\subsection{Wykorzystane funkcjonalności}
		\begin{center}
			\begin{tabular}{|c|c|}
				\hline 
				\textbf{Funkcjonalność} & \textbf{Osoba za nią odpowiedzialna} \\ 
				\hline 
				GPIO (joystick) & Konrad Jaworski \\ 
				\hline 
				Akcelerometr & Konrad Jaworski \\ 
				\hline 
				Głośnik & Konrad Jaworski \\ 
				\hline 
				Timer & Przemysław Rudowicz \\ 
				\hline 
				OLED & Przemysław Rudowicz \\ 
				\hline 
				SSP/SPI & Przemysław Rudowicz \\ 
				\hline 
				Czujnik światła & Jakub Plich \\ 
				\hline 
				pca9532 & Jakub Plich \\ 
				\hline 
				I\textsuperscript{2}C & Jakub Plich \\ 
				\hline 
				Rotacyjny przełącznik kwadraturowy & Jakub Plich\\
				\hline
			\end{tabular} 	
		\end{center}
	}


	\subsection{Podział obowiązków}
\begin{center}
			\begin{tabular}{|c|c|}
			\hline 
			\textbf{Imię i nazwisko} & \textbf{Procentowy udział w pracy} \\ 
			\hline 
			Konrad Jaworski & 33\% \\ 
			\hline 
			Przemysław Rudowicz & 34\% \\ 
			\hline 
			Jakub Plich & 33\% \\ 
			\hline 
		\end{tabular} 
\end{center}
\newpage

	\section{Opis działania programu}
	
		\subsection{Instrukcja użytkownika}
		
		\subsection{Opis algorytmu}
		
	\section{Funkcjonalności}
		\subsection{GPIO}
		GPIO (oznacza general-purpose input/output) - interfejs wejścia/wyjścia ogólnego przeznaczenia. Należy ustawić kierunki wejścia/wyjścia pinów GPIO (0 - gdy chcemy skonfigurować pin jako wejście, lub 1 - jako wyjście).
		
		\subsubsection{Głośnik}
		Głośnik jest obsługiwany przy pomocy pinów GPIO.  Jako, że głośnik nie będzie wysyłał danych, piny ustawiamy na wyjście. W tym celu ustawiamy wartość 1 w rejestrach FIODIR0 i FIODIR2 w miejscach odpowiadających pinom głośnika (każdy bit rejestru odpowiada jednemu pinowi GPIO, każdy port GPIO ma swój rejestr FIODIR). A więc 1 należy ustawić na 28, 27, 26 bicie FIODIR0 i 13 bicie FIODIR2.
		
		
		Wzmacniacz analogowy LM4811, który znajduje się na płytce LPCXpresso Base Board potrzebuje następujących pinów z mikrokontrolera:
		\begin{itemize}
			\item CLK
			\item UP/DN
			\item SHUTDN
			\item VIN1/VIN2
		\end{itemize}
	
		\medskip
		Ze specyfikacji LM8411 \cite{LM4811} dowiadujemy się, że piny CLK (CLOCK) oraz UP/DN
		są odpowiedzialne za sterowanie głośnością brzęczyka.
		
		
		Pin SHUTDN aktywuje funkcję zmniejszającą pobór prądu przez wzmacniacz (Nie korzystamy z tej funkcji).
		
		
		Piny VIN1/VIN2 odpowiadają za generację sygnału wprawiającego
		membranę brzęczyka w drgania (generowanie dźwięków).\\
		
		
		\begin{center}
			Sposób połączenie pinów wzmacniacza analogowego do pinów GPIO:\smallskip

			\begin{tabular}{|c|c|}
				\hline 
				\textbf{Piny LM4811} & \textbf{Piny GPIO} \\ 
				\hline 
				CLK & P0.27 \\ 
				\hline 
				UP/DN & P0.28 \\ 
				\hline 
				SHUTDN & P2.13 \\ 
				\hline 
				VIN1/VIN2 & P0.26 \\ 
				\hline 
			\end{tabular} 
		\end{center}
	
		Podczas inicjalizacji głośnika czyszczone są wartość na pinach P0.27, P0.28, P2.13 (ustawiamy 1 w rejestrach FIOCLR dla portu 0 i 2 w miejscach odpowiadających wymienionym pinom).
		
		
		Generowanie dźwięku przez brzęczyka odbywa się poprzez podawaniu zmiennego napięcia na pin P0.26 tak aby
		wprowadzić membranę brzęczyka w drgania. Pozwala to na generowanie prostych nut. 
		
		
		Aby zagrać nutę 'C', należy wprowadzić membranę brzęczka w drgania o częstotliwości $f=262Hz$. A więc okres drań $T=\frac{1}{f} = 3816 \mu s$. Stąd na pinie P0.26 należy ustawić stan wysoki przez czas równy $\frac{T}{2} = 1908\mu s$ oraz stan niski analogicznie przez  $\frac{T}{2}$. Cykl należy powtarzać w zależności od tego jak długo chcemy odtwarzać dźwięk. Do ustawiania stanów wysokich i niskich na pinach GPIO używamy rejestru FIOSET i FIOCLR. Za
		generowanie dźwięku odpowiada pin P0.26. Analogicznie postępujemy w przypadku innych nut.
		
		W celu ustawienia stanu wysokiego na pinie P0.26 należy ustawić 1 na 26 bicie rejestru FIOSET (ustawianie zera na tym rejestrze nie ustawia stanu niskiego). Aby odwołać stan wysoki należy wpisać 1 na 26 bicie rejestru FIOCLR.
		\subsubsection{Joystick}
		
		\subsection{Akcelerometr}
	
		
		\subsection{Timer}
		
		\subsection{OLED}
		
		\subsection{SSP/SPI}
		
		\subsection{Czujnik światła}
		
		\subsection{PCA9532}
		
		\subsection{I\textsuperscript{2}C}
		
		\subsection{}
	\section{Analiza FMEA}
	

	
	\hspace*{-85pt}\begin{tabular}{|c|c|c|c|c|c|}
		\hline 
		Ryzyko & Prawdopodobieństwo & Znaczenie & (Samo)Wykrywalność & Iloczyn & Reakcja \\ 
		\hline 
		Uszkodzenie joysticka &  &  &  &  &  \\ 
		\hline 
	\end{tabular} 


\begin{thebibliography}{abcd}
	\bibitem{LM4811}{
		\textit{LM4811
			Dual 105mW Headphone Amplifier with Digital Volume
			Control and Shutdown Mode Datasheet},
		{December 2002},
		\textbf{National Semiconductor}
	}		
\end{thebibliography}

\end{document}
