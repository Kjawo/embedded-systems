\documentclass{classrep}
\usepackage[utf8]{inputenc}
\frenchspacing

\usepackage{graphicx}
\usepackage{subfig}
\usepackage{float}
\usepackage[usenames,dvipsnames]{color}
\usepackage[hidelinks]{hyperref}

\usepackage{amsmath, amssymb, mathtools}

\usepackage{hyperref}

\usepackage{fancyhdr, lastpage}
\pagestyle{fancyplain}
\fancyhf{}
\renewcommand{\headrulewidth}{0pt}
\cfoot{\thepage\ / \pageref*{LastPage}}


\studycycle{Informatyka, studia dzienne, I st.}
\coursesemester{IV}

\coursename{Systemy wbudowane}
\courseyear{2018/2019}

\courseteacher{dr inż. Michał Morawski}
\coursegroup{środa, 12:15}

\author{%
	\textbf{Grupa D07}\\\\
	\studentinfo[216879@edu.p.lodz.pl]{Przemysław Rudowicz}{216879} - lider\\
	\studentinfo[216782@edu.p.lodz.pl]{Konrad Jaworski}{216782}\\
	\studentinfo[216866@edu.p.lodz.pl]{Jakub Plich}{216866}%
}

\title{Dokumentacja projektu gry Snake\\
		LPC1768/9}
\begin{document}
	\thispagestyle{fancyplain}
	\maketitle
	

	
	\newpage
	\tableofcontents
	\newpage
	
	\section{Lista wykorzystanych funkcjonalności}
	{
		Wykorzystane funkcjonalności:\\
\begin{center}
	\begin{tabular}{|c|c|}
	\hline 
	Funkcjonalność & Osoba za nią odpowiedzialna \\ 
	\hline 
	f1 & Konrad Jaworski \\ 
	\hline 
	f2 & Przemysław Rudowicz \\ 
	\hline 
	f3 & Jakub Plich \\ 
	\hline 
\end{tabular} 
\end{center}
	}
	    

\end{document}
