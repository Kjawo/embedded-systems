\documentclass{classrep}
\usepackage[utf8]{inputenc}
\frenchspacing

\usepackage{graphicx}
\usepackage{subfig}
\usepackage{float}
\usepackage[usenames,dvipsnames]{color}
\usepackage[hidelinks]{hyperref}

\usepackage{adjustbox}

\usepackage{amsmath, amssymb, mathtools}

\usepackage{hyperref}

\usepackage{fancyhdr, lastpage}

\usepackage{array}

\pagestyle{fancyplain}
\fancyhf{}
\renewcommand{\headrulewidth}{0pt}
\cfoot{\thepage\ / \pageref*{LastPage}}


\studycycle{Informatyka, studia dzienne, I st.}
\coursesemester{IV}

\coursename{Systemy wbudowane}
\courseyear{2018/2019}

\courseteacher{dr inż. Michał Morawski}
\coursegroup{środa, 12:15}

\author{%
	\textbf{Grupa D07}\\\\
	\studentinfo[216879@edu.p.lodz.pl]{Przemysław Rudowicz}{216879} - lider\\
	\studentinfo[216782@edu.p.lodz.pl]{Konrad Jaworski}{216782}\\
	\studentinfo[216866@edu.p.lodz.pl]{Jakub Plich}{216866}%
}

\title{Dokumentacja projektu gry Snake\\
		LPC1768/9}
\begin{document}
	\thispagestyle{fancyplain}
	\maketitle
	

	
	\newpage
	\tableofcontents
	\newpage
	
	\section{Podział obowiązków}
	{
	\subsection{Wykorzystane funkcjonalności}
		\begin{center}
			\begin{tabular}{|c|c|}
				\hline 
				\textbf{Funkcjonalność} & \textbf{Osoba za nią odpowiedzialna} \\ 
				\hline 
				GPIO (joystick) & Konrad Jaworski \\ 
				\hline 
				Akcelerometr & Konrad Jaworski \\ 
				\hline 
				Głośnik & Konrad Jaworski \\ 
				\hline 
				Timer & Przemysław Rudowicz \\ 
				\hline 
				OLED & Przemysław Rudowicz \\ 
				\hline 
				SSP/SPI & Przemysław Rudowicz \\ 
				\hline 
				Czujnik światła & Jakub Plich \\ 
				\hline 
				pca9532 & Jakub Plich \\ 
				\hline 
				I\textsuperscript{2}C & Jakub Plich \\ 
				\hline 
				Rotacyjny przełącznik kwadraturowy & Jakub Plich\\
				\hline
			\end{tabular} 	
		\end{center}
	}


	\subsection{Podział obowiązków}
\begin{center}
			\begin{tabular}{|c|c|}
			\hline 
			\textbf{Imię i nazwisko} & \textbf{Procentowy udział w pracy} \\ 
			\hline 
			Konrad Jaworski & 33\% \\ 
			\hline 
			Przemysław Rudowicz & 34\% \\ 
			\hline 
			Jakub Plich & 33\% \\ 
			\hline 
		\end{tabular} 
\end{center}
\newpage

	\section{Opis działania programu}
	
		\subsection{Instrukcja użytkownika}
		Po podłączeniu płytki do zasilania na wyświetlaczu OLED zostanie narysowany wąż (reprezentowany przez ciąg pikseli tworzący linię łamaną) oraz owoc reprezentowany przez grupę 4 pikseli (kwadrat 2x2 w lewej, górnej części ekranu).\\
		
		
		Aby rozpocząć grę należy wcisnąć przycisk z prawej strony joysticka.\\
		
		
		Do zmiany kierunku poruszania się węża można użyć joysticka lub akcelerometru (przechylać płytkę).
		
		
		Celem gry jest zebranie/zjedzenie jak największej liczy owoców. Dokonuje się to poprzez zderzenie się głowy węża oraz ciała owocu. Aktualny wynik gracza (liczba zjedzonych owoców) jest przedstawiany w postaci binarnej na diodach LED expandera.
		
		
		Gracz przegrywa w momencie, gdy głowa węża uderzy w ścianę lub w ciało węża (również, gdy gracz spróbuje 'cofnąć się' - zmienić kierunek poruszania się na przeciwny do obecnego).
		
		
		Gracz może kontrolować szybkość poruszania się węża przy pomocy rotacyjnego przełącznika kwadraturowego (obrót w lewo zwiększa prędkość, a w prawo zmniejsza).
		
		
		Po zakryciu czujnika światła, kolory na ekranie zostają odwrócone.
		
		
		Aby zagrać kolejny raz należy ponownie wcisnąć przycisk z prawej strony joysticka.
		
		
		
		\subsection{Opis algorytmu}
		Program został napisany w języku C. Wąż porusza się poprzez usuwanie ostatniej części swojego ciała i dodawanie nowego punktu na pozycji (obok głowy) wskazanej przez kierunek w którym ma się poruszać. Kierunek jest ustalany przez gracza poprzez wybranie go joystickiem lub przechylenie płytki w wybraną stronę.
		
		W głównej pętli funkcji main() sprawdzamy stan rotacyjnego przełącznika kwadraturowego i jeżeli program wykryje jego obrót, następuje inkrementacja lub dekrementacja zmiennej odpowiedzialnej za czas przestoju węża. Następnie pobierany jest stan akcelerometru i joysticka. Jeżeli stan joysticka zgadza się z wybranymi kierunkami (lewo, prawo, góra, dół) następuje przypisanie tego kierunku do kierunku poruszania się węża. Jeżeli wartości oczytane z akcelerometru przekroczą zadaną wartość, również nastąpi przypisanie do kierunku poruszania się węża, kierunku, w którym płytka została przechylona. Jeżeli wąż nie jest w stanie zablokowanym (nie uderzył wcześniej w ściany lub w samego siebie), wywoływana jest funkcja odpowiedzialna za poruszanie się węża, a następnie rysowanie go i sprawdzanie kolizji.
		Jeżeli program wykryje przysłonięcie czujnika światła, następuje odwrócenie kolorów rysowanych obiektów.
		Sprawdzany jest również stan wciśnięcia przycisku odpowiedzialnego za rozpoczynanie nowej tury gry.
		
	\section{Opis działania wykonanego sprzętu}
	Nie było wykonanego sprzętu.
	\section{Funkcjonalności}
		\subsection{GPIO}
		GPIO (oznacza general-purpose input/output) - interfejs wejścia/wyjścia ogólnego przeznaczenia. Należy ustawić kierunki wejścia/wyjścia pinów GPIO (0 - gdy chcemy skonfigurować pin jako wejście, lub 1 - jako wyjście).
		
		\subsubsection{Głośnik}
		Głośnik jest obsługiwany przy pomocy pinów GPIO.  Jako, że głośnik nie będzie wysyłał danych, piny ustawiamy na wyjście. W tym celu ustawiamy wartość 1 w rejestrach FIODIR0 i FIODIR2 w miejscach odpowiadających pinom głośnika (każdy bit rejestru odpowiada jednemu pinowi GPIO, każdy port GPIO ma swój rejestr FIODIR). A więc 1 należy ustawić na 28, 27, 26 bicie FIODIR0 i 13 bicie FIODIR2.
		
		
		Wzmacniacz analogowy LM4811, który znajduje się na płytce LPCXpresso Base Board potrzebuje następujących pinów z mikrokontrolera:
		\begin{itemize}
			\item CLK
			\item UP/DN
			\item SHUTDN
			\item VIN1/VIN2
		\end{itemize}
	
		\medskip
		Ze specyfikacji LM8411 \cite{LM4811} dowiadujemy się, że piny CLK (CLOCK) oraz UP/DN
		są odpowiedzialne za sterowanie głośnością brzęczyka.
		
		
		Pin SHUTDN aktywuje funkcję zmniejszającą pobór prądu przez wzmacniacz (Nie korzystamy z tej funkcji).
		
		
		Piny VIN1/VIN2 odpowiadają za generację sygnału wprawiającego
		membranę brzęczyka w drgania (generowanie dźwięków).\\
		
		
		\begin{center}
			Sposób połączenie pinów wzmacniacza analogowego do pinów GPIO:\smallskip

			\begin{tabular}{|c|c|}
				\hline 
				\textbf{Piny LM4811} & \textbf{Piny GPIO} \\ 
				\hline 
				CLK & P0.27 \\ 
				\hline 
				UP/DN & P0.28 \\ 
				\hline 
				SHUTDN & P2.13 \\ 
				\hline 
				VIN1/VIN2 & P0.26 \\ 
				\hline 
			\end{tabular} 
		\end{center}
	
		Podczas inicjalizacji głośnika czyszczone są wartość na pinach P0.27, P0.28, P2.13 (ustawiamy 1 w rejestrach FIOCLR dla portu 0 i 2 w miejscach odpowiadających wymienionym pinom).
		
		
		Generowanie dźwięku przez brzęczyka odbywa się poprzez podawaniu zmiennego napięcia na pin P0.26 tak aby
		wprowadzić membranę brzęczyka w drgania. Pozwala to na generowanie prostych nut. 
		
		
		Aby zagrać nutę 'C', należy wprowadzić membranę brzęczka w drgania o częstotliwości $f=262Hz$. A więc okres drań $T=\frac{1}{f} = 3816 \mu s$. Stąd na pinie P0.26 należy ustawić stan wysoki przez czas równy $\frac{T}{2} = 1908\mu s$ oraz stan niski analogicznie przez  $\frac{T}{2}$. Cykl należy powtarzać w zależności od tego jak długo chcemy odtwarzać dźwięk. Do ustawiania stanów wysokich i niskich na pinach GPIO używamy rejestru FIOSET i FIOCLR. Za
		generowanie dźwięku odpowiada pin P0.26. Analogicznie postępujemy w przypadku innych nut.
		
		W celu ustawienia stanu wysokiego na pinie P0.26 należy ustawić 1 na 26 bicie rejestru FIOSET (ustawianie zera na tym rejestrze nie ustawia stanu niskiego). Aby odwołać stan wysoki należy wpisać 1 na 26 bicie rejestru FIOCLR.
		\subsubsection{Joystick}
		Joystick również jest obsługiwany przy pomocy pinów GPIO. Natomiast w przeciwieństwie do głośnika, joystick wysyła dane do mikrokontrolera, a więc podczas jego inicjalizacji ustawiamy wszystkie piny na wejście.
		
		W tym celu ustawiamy wartość '0' na 15, 16 i 17 bicie rejestru FIODIR0 oraz na 3 i 4 bicie rejestru FIODIR2. 
		
\begin{center}
			\begin{tabular}{|c|c|c|}
			\hline 
			\textbf{Pozycja joysticka} & \textbf{wartość} & \textbf{Piny GPIO} \\ 
			\hline 
			JOYSTICK\_CENTER & 0x01 & P0.17 \\ 
			\hline 
			JOYSTICK\_UP & 0x02 & P2.3 \\ 
			\hline 
			JOYSTICK\_DOWN & 0x04 & P0.15 \\ 
			\hline 
			JOYSTICK\_LEFT & 0x08 & P2.4 \\ 
			\hline 
			JOYSTICK\_RIGHT & 0x10 & P0.16 \\ 
			\hline 
		\end{tabular} 
\end{center}
		
		Stany podłączonych pinów odpowiadają stanom wciśnięcia joysticka (odpowiednio tak jak w tabeli powyżej).
		W celu odczytania stanu joysticka sprawdzane są wartości na kolejnych pinach (odpowiednio tak jak w tabeli powyżej) i jeżeli jego wartość to '0', zmienna przechowująca stan joysticka przyjmuje wartość koniunkcji bitowej tego stanu i odpowiadającej wartości (patrz tabela powyżej) przypisanej do pozycji joysticka. 
		\subsection{Akcelerometr}
	Użyty przez nas akcelerometr MMA7455L \cite{MMA7455L} jest inicjalizowany (za pomocą I\textsuperscript{2}C (adres urządzenia - 0x1D)) przez zdefiniowanie trybu pomiaru i czułości. Dokonujemy tego poprzez wpisanie do rejestru MCTL (Mode Control o adresie 0x16) wartości 1 na bicie zerowym (odpowiada za ustawienie czułości na 2g) oraz wartości 1 na bicie drugim (odpowiada za ustawienie trybu pomiarowego (Measurement Mode)).
	
\begin{center}
		\resizebox{\textwidth}{!}{%
	\begin{tabular}{|c|c|c|c|c|c|}
		\hline 
		Adres & Nazwa rejestru & Bit 3 (GLVL[1]) & Bit 2 (GLVL[0]) & Bit 1 (MOD[1]) & Bit 0 (MOD[0]) \\ 
		\hline 
		0x16 & MCTL & 0 & 1 & 0 & 1 \\ 
		\hline 
	\end{tabular} 
	}
\end{center}

		Aby odczytać wartości wskazań akcelerometru najpierw zostaje sprawdzona wartość z rejestru przechowujące informację o statusie akcelerometru (rejestr o adresie 0x09). Na bicie zerowym tego rejestru (DRDY) przechowywana jest informacja, czy dane są gotowe do odczytania.
\begin{center}
			\begin{tabular}{|c|c|}
			\hline 
			\vtop{\hbox{\strut \textbf{Wartość na bicie DRDY}}\hbox{\strut \textbf{rejestru 0x09}}} & \textbf{Znaczenie} \\ 
			\hline 
			1 & dane są gotowe do odczytania \\ 
			\hline 
			0 & dane nie są gotowe do odczytania \\ 
			\hline 
		\end{tabular} 
\end{center}
		
		Jeżeli dane są gotowe do odczytania, z rejestru XOUT8 (o adresie 0x06) pobierane jest 8 bitów, które opisują składową X wektora przyspieszenia. Następnie, kolejno odczytywane są wartości z rejestrów YOUT8 (0x07) oraz ZOUT8 (0x08), które opisują odpowiednio składowe Y i Z. 
		
\begin{center}
			\begin{tabular}{|c|c|}
			\hline 
			\textbf{Adres rejestru} & \textbf{Nazwa rejestru} \\ 
			\hline 
			0x06 & XOUT8 \\ 
			\hline 
			0x07 & YOUT8 \\ 
			\hline 
			0x08 &  ZOUT8\\ 
			\hline 
		\end{tabular} 
\end{center}
			
		\subsection{Timer}
		Urządzenie wyposażone jest w 4 timery - timer0, timer1, timer2, timer3. W naszym projekcie wykorzystujemy 2 z nich - timer0 oraz timer1. 
		\newline Najważniejsze rejestry, używane do konfiguracji Timera to:
		\newline \textbullet Rejestr IR - Zawiera informacje, czy Timer oczekuje na obsługe przerwania. Może zostać on wyzerowany w celu usunięcia oczekujących przerwań.
		\newline \textbullet Rejestr TCR - wyzerowanie spowoduje restart Timera
		\newline \textbullet Rejestr PC - jest inkrementowany co każdy cykl PCLK
		\newline \textbullet Rejestr PR - zawiera wartość, co ile cykli PCLK następuje inkrementacja w rejestrze TC. Gdy wartość w PC jest równa PR nastepuje inkrementacja w TC i zerowanie PC.
		\newline \textbullet Rejestr TC - licznik Timera, jest inkrementowany co określoną liczbe cykli. Z tego rejestru pobieramy czas
		\newline \textbullet Rejestr MR0 - jeśli wartość z TC jest równa MR0 wywołane zostaje przerwanie.
		\newline \textbullet Rejestr MCR - konfiguracja dotycząca przerwań.
		
		
		Podczas inicjalizacji Timer korzysta z dwóch struktur, które przechowują ustawienia:
		\newline
		\textbullet Pierwsza stuktura pozwala na ustawienie trybu Timera. Dostepne tryby to tryb absolutny - Timer zlicza "tyknięcia" zegara i zwraca liczbe tyknięć oraz tryb preskalowany - wtedy Timer zwraca wartość w jednostce czasu, oraz wartość preskalowania - 1000 dla milisekund.
		\newline
		\textbullet Druga struktura zawiera opcje dotyczące rejestru MCR, takie  jak:
			\newline \textbullet Wybór kanału dopasowania (od 0 do 3)
			\newline \textbullet Stwiedzenie, czy po dopasowaniu ma wystąpić przerwanie (ENABLE lub DISABLE)
			\newline \textbullet Stwierdzenie, czy po dopasowaniu timer ma się zatrzymać (ENABLE lub DISABLE)
			\newline \textbullet Stwierdzenie, czy po dopasowaniu timer ma się zresetować  
			\newline \textbullet Wybór, co ma się stać na lini obsługującej przerwania. Możliwości to nierobienie niczego, zmiana stanu na niski, zmiana stanu na wysoki, zmiana stanu w ogóle.
			
		 Timer jest inicjalizowany poprzez ustawienie odpowiednich rejestrów. W naszym programie Timer0 jest inicjalizowany następującymi wartościami:
		\newline \textendash Do rejestru PR wpisanie wartości dotyczącej preskalowania do milisekund (1000)
		\newline \textendash Do rejestru MR0 czas, po jakim na następować przerwanie
		\newline \textendash W rejestrze MCR - bit 0 ustawiony na 1 - włączenie przerwań oraz bit 2 ustawiony na 1 - po przerwaniu stop.
		\newline \textendash Do rejestru EMR - 0 nie rób nic po przerwaniu.
		\newline \textendash Końcową fazą inicjalizacji jest ustawienie w rejestrze TCR na 0 bicie wartości 1. Implikuje to restart Timera.	
		\newline Inicjalizacja Timer'a1 jest bardzo podobna do inicjalizacji wcześniejszego. Różnicą jest, że w tym przypadku timer nie korzysta z przerwań. Zatem w rejestrze MCR ustawione zostanie 0. Reszta operacji przebiega tak samo jak podczas inicjalizacji Timer0.
		\newline \newline
		Zczytanie wartości z Timer1 sprowadza się do odczytania wartości rejestru TC.
		
		\subsection{OLED}
		Wyświetlacz OLED, z którego korzystamy to OEL Display Module, model UG-9664HSWAG01. 
		Podstawowa Specyfikacja:
		\newline \textendash Rozdzielczość:  $96\times 64$
		\newline \textendash Kolor wyświetlacza: Monochromatyczny (biały)
		
		Inicjalizacja wyświetlacza rozpoczyna się od ustawienia kierunku wyjścia na portach GPIO nr 2 i 0.
		
		
		 \begin{tabular}{|c|c|c|}
		 	\hline 
		 	\textbf{Nr portu} & \textbf{Bity, których kierunek ustawiamy} & \textbf{Kierunek}  \\ 
		 	\hline 
		 	2 & bit 1 & Wyjście \\ 
		 	\hline 
		 	2 & bit 7 & Wyjście  \\ 
		 	\hline 
		 	0 & bit 6 & Wyjście \\
		 	\hline
		 	 \end{tabular} 
		 	
		 	Następnie należy upewnić się, że zasilanie wyświetlacza jest wyłączone poprzez wyczyszczenie bitu 1 na porcie nr 2 GPIO. Z punktu widzenia wyświetlacza, realizowane jest to przez pin MPU nr 9. Gdy sygnał na tym pinie jest niski, realizowany jest reset zasilania wyświetlacza. \cite{SAS1-6020-A}
		 	
		 	Kolejny krokiem jest sekwencja poleceń rekomendowanych do wykonania przez producenta urządzenia. Odbywa się to przez interfejs SSP. Adres wyświetlacza OLED w interfejsie SSP wynosi $($0x40000000UL$)$ $+$  $($0x30000$)$.
		 	Ostatnim punktem inicjalizacji wyświetlacza jest już włączenie zasilania. Odbywa się to poprzez wstawienie 1 na bicie nr. 1 na porcie nr. 2 GPIO.
		 	
		 	
		 	Wyświetlenie obiektu na ekranie sprowadza się do zamalowania kolorem przeciwnym do tła pikseli. Przesyłanie informacji, który piksel ma zostać zamalowany odbywa się za pomocą interfejsu SPI. Minimalna wartość do przesłania to 8 bitów. Wysłanie jednego piksela w ten sposób powoduje, że należy pozostałe 7 ustawić na czarno. Z tego powodu wysyłanie na ekran jest wolniejsze i powiązane z szybkością magistrali SSP.
		 	
		 	Wysyłanie pikselu na ekran rozpoczyna się od ustalenia, w której stronie znajduje się punkt. Należy również obliczyć dolny oraz górny adres kolumny oraz wartość maski, która to jest resztą z dzielenia składowej y piksela przez 8. Maska jest pozycją bitu, na której znajduje się piksel który ma zostać zamalowany na biało. Kolejnym krokiem jest wysłanie przez SSP numery strony, dolnego i górnego adresu kolumny. Ostatecznie przez SSP wysyłane są bity z informacją, który piksel zamalować. 
		 	
			
			
			
		
		
		\subsection{SSP/SPI}
		SPI (Serial Peripheral Interface ) \textendash to szeregowy interfejs urządzeń peryferyjnych, umożliwiający komunikację między mikroprocesorem a urządzeniami peryferyjnymi. Umożliwia synchroniczną, szeregową, pełnodupleksową komunikację oraz programowalną długość przesyłanych danych.
		Płytka wyposażona jest również w dwa kontrolery SSP (Synchronous Serial Port ), SSP0 i SSP1 – kontrolery umożliwiające operacje na SPI, SSI oraz magistrali Microwire. Posiadają one kolejkowanie FIFO, obsługę 	wielu protokołów oraz możliwość używania z wykorzystaniem General Purpose DMA.
		
		Interfejs SPI znajduję się w LPC1768/9, jednak jest to przestarzałe rozwiązanie. Jest on zaimplementowany w urządzeniu, by łatwiej przenieść oprogoramowanie ze starszych modeli, które nie miały możliwości użycia SSP. Jako, że interfejs SSP jest nowszym i bardziej wszechstronnym rozwiązaniem, skorzystaliśmy właśnie z niego, co również zaleca dokumentacja. W projekcie wykorzystujemy jeden z dwóch interfejsów - SPP1. Jego adres bazowy to 0x4003 0000, jest podłączony do slave group 0 (APB0).
		
		 	Komunikacja w trybia SPI odbywa się synchronicznie za pomocą czterech lini:
		 \textbullet MOSI (Master Output Slave Input) \textendash dane wysyłane do urządzenia peryferyjnego
		 \textbullet MISO (Master Input Slave Output) \textendash odczyt danych z urządzenia peryferyjnego 
		 \textbullet SCLK (Serial CLocK) - sygnał zegarowy zapewniający synchroniczność
		 \textbullet SS (Slave Select) - służy do wyboru urządzenia peryferyjnego
		 
		 Co prawda możliwa jest obsługa wielu połączeń master-slave, jednak tylko jeden master i tylko jeden slave mogą się komunikować podczas przesyłania danych. Przesył danych przebiega w trybie full-duplex, wykorzystywane są ramki o długości 8 lub 16 bitów. 
		 
		 Inicjalizację zasilania zapewnia rejestr PCONP. By zasilić SSP1 należy ustawić w tym rejestrze bit o nr. 10 na wartość 1. Następuje konfiguracja SSP, ustawienie bitów rejestrów CR0 i CR1.
		 
		 \begin{center}
		 	Tabela przedstawiająca konfiguracje na rejestrze CR0
		 
		 
		 \begin{tabular}{|c|c|c|c|c|}
		 	\hline 
		 Bit 3:0  & Bit 5:3  & Bit 6  & Bit 7  &  \\ 
		 	\hline 
		 DSS & FRF  & CPOL  & CPHA & SCR \\ 
		 	\hline 
		 \end{tabular} 
	 \end{center}
	 
	 
	 \textbullet DSS  - określenie rozmiaru danych w ramce
	 \textbullet FRF  - tryb pracy
	 \textbullet CPOL - wybór polaryzacji sygnału
	 \textbullet CPHA - wybór fazy próbkowania
	 \textbullet SCR  - podzielnik sygnału preskalera, pozwala określić prędkość transmisji 
		  
			 
		\begin{center}
			Tabela przedstawiająca konfiguracje na rejestrze CR1
			
			
			\begin{tabular}{|c|c|c|c|}
				\hline 
				Bit 0  & Bit 1  & Bit 2  & Bit 3  \\ 
				\hline 
				LBM & SSE  & MS  & SOD \\ 
				\hline 
			\end{tabular} 
		\end{center}
	
	\textbullet LBM - ustawienie do debugowania
	\textbullet SSE - włączenie SSP
	\textbullet MS  - określenie trybu pracy. 0 dla master, 1 dla slave
	\textbullet SOD - dla 1 występuje blokada mastera do wysyłania danych linią MISO
	
	
	Aby poprawnie przesyłać dane należy określić prędkość transmisji, poprzes ustawienie jej w rejestrze CPSR (Clock Prescale Register).
	
	
	Po zainicjalizowaniu, aby rozpocząć przesyłanie danych, należy je przesłać do rejestru danych DR (jeśli bit TNF w rejestrze statusu SR jest ustawiony). Jeżeli bit BSY w rejestrze SR jest ustawiony na 0, co oznacza, że kontroler nie jest zajęty, rozpoczyna się nadawanie.
		
		\subsection{Czujnik światła}
Czujnik światła jest urządzeniem peryferyjnym przyłączonym do płytki magistralą I\textsuperscript2C. Program wkorzystuje odczytane natężenie światła do odwrócenia kolorów na 			wyświetlaczu w momencie w którym odczytana wartość natężenia światła będzie mniejsza niż 25 luksów. Uruchomienie czyjnika światła polega na przesłaniu przez	
I2C pod adres (0x44) kolejno wartości (0x00) oraz (1<<7) . To powoduje ustawienie wartości 1 na 7 bicie rejestru Command Register(0x00) i w konsekwencji 	uruchomienie przetwornika analogowo-cyfrowego w czujniku. Zakres odczytu czujnika jest domyślny i wynosi od 0 do 1000 luksów. Odczyt wartości pomiaru czujnika wymaga 				odczytania zawartości dwóch rejestrów:  LSB-Sensor - zawiera dolny bajt ostatniego odczytu sensora(adres 0x04), MSB-Sensor - zawiera górny bajt ostatniego odczytu 					sensora(adres 0x05). Wynik wyrażony w luksach jest obliczany z następującego wzoru: E = 973 * odczytana-wartość /  (1<<16)
\subsection{PCA9532}
Expander PCA9532 wyposażony w 16 diód LED jest wkorzystywany do reprezentacji wyniku w danym momencie gry. Zapalone diody przedstawiają wynik w postaci binarnej.
W celu zapalenia odpowiednich diód należy ustalić 16 bitową maskie w której wartości 1 oznaczają zapaloną diodę. Następnie przez I2C pod adres rejestru kontrolnego(0x60)  				zostaje wysłany bajt kontrolny. Trzeci  bit bajtu kontorlnego onacza flagę inkrementacji która zwiększa o 1 adres podany w pozostałych 4 bitach po każdym przesłanym bajcie. 				Potem przesłane zostają 4 bajty które zostają kolejno wpisane do 4  8-bitowych rejestrów(LS0,LS1,LS2,LS3) w których każde 2 bity odpowiadają jednej diodzie. Program korzysta 				jedynie ze stanów OFF(00) oraz ON(01).
\subsection{I\textsuperscript{2}C}
I2C (Inter-Integrated Circuit) to szeregowa, dwukierunkowa magistrala do przesyłania danych. Program wykorzystuje ją do komunikacji z PCA9532, czujnikiem światła oraz 					akcelerometrem.
Początek inicjalizacji magistrali zaczyna się od konfiguracji pinów które będą pełniły fukcjię lini SCL(linia zegara) i SDA(linia danych). Dla interfejsu I2C2 są one ustawione 				odpowiednio na pinach P0[11] oraz P0[10]. W tym celu ustawiamy w rejestrze PINSEL0 wartosci 1 i 0 kolejno dla bitów 21 i 20(P0.10 SDA2) oraz wartość 1 i 0 dla bitów 23 i 22(P0.11 		SCL). Następnie następuje włączenie zasilania dla I2C poprzez ustawienie wartości 1 na 26 bicie rejestru PCONP(Power Control for Peripherals Register). Ustawiamy dzielnik zegara 			PCLK na 2 ustawiając w rejestrze PCLKSEL 20 i 21 bit na wartości kolejno 0 i 1. Następnie należy ustawić wartość rejestrów I2SCLH i I2SCLL na żądaną ilość cykli zegara PCLK. Obie			 	te wartości są sobie równe.(...)
 
Na koniec występuje ustawienie na wartość 1 bitów 2(flaga AA), 5(flaga STA) i 6(flaga I2EN) rejestru CONCLR.
Na koniec należy w rejestrze I2CONSET ustawiamy wartość 1 na 6 bicie aby włączyć interfejs I2C2. 
\subsection{Rotacyjny przełącznik kwadraturowy}
Rotacyjny przełącznik kwadraturowy jest wykorzystywany w naszym programie do regulacji prędkości poruszania się "węża". Obrót przełącznika w prawo zmniejsza prędkość, a obór w lewo zwiększa. Inicjalizacja ogranicza się do ustawienia na porcie 0 GPIO wartości 0 na 24 i 25 bicie rejestru FIODIR(zerowy i pierwszy bit rejestru FIO0DIR3). Wartość 0 oznacza że pin jest ustawiony na wejście. Program odczytuje inforamcję o 4 stanach(są ustalone na podstawie położenia przełącznika). W zależności od kolejności występowania stanów ustalany jest kierunek ruchu przełącznika.
	\section{Analiza FMEA}
	

	\resizebox{\textwidth}{!}{%
	\hspace*{-85pt}\begin{tabular}{| m{3cm}| m{4cm} |m{3cm}|m{8em}|m{3cm}|m{8em}|}
		\hline 
		Ryzyko & Prawdopodobieństwo & Znaczenie & (Samo)Wykrywalność & Iloczyn & Reakcja \\ 
		\hline 
		Uszkodzenie joysticka & 0.05 & 8 (Znaczne) & 5 (średnia), wykrywane, jeżeli program odczytuje z  joysticka tą samą wartość przez 50 iteracji  & 0.5 &  Blokada zczytywania wartości z joysticka, przejście na sterowanie akcelerometrem  \\ 
		\hline 
		Uszkodzenie akcelerometru & 0.1 & 5 (Średnie) & 3 (średnia), wykrycie po wielokrotnym odczycie spoza zakresu urządzenia & 0.3 & Wyłączenie sterowania przy pomocy akcelerometru, przejście na\newline sterowanie tylko joystickem \\
		\hline
		Awaria głośnika & 0.2 & 2 (Nieznaczące) & 1(łatwa), organoleptyczna wykrywalność - brak dźwięku implikuje zepsucie głośnika & 0.1 & Brak reakcji \\
		\hline
		Awaria wyświetlacza OLED & 0.2 & 10 (Krytyczne) & brak implementacji & 10 & Brak reakcji, zepsucie wyświetlacza sprawia, że gracz nie widzi, czym steruje, gra traci sens \\
		\hline
	\end{tabular} 
}


\begin{thebibliography}{abcd}
	\bibitem{MMA7455L}{
		\textit{$\pm$2g/$\pm$24g/$\pm$28g Three Axis Low-g
			Digital Output Accelerometer},
		{Rev 8, 07/2009},
		\textbf{Freescale Semiconductor}
	}	

	\bibitem{LM4811}{
	\textit{LM4811
		Dual 105mW Headphone Amplifier with Digital Volume
		Control and Shutdown Mode Datasheet},
	{December 2002},
	\textbf{National Semiconductor}
}
		\bibitem{SAS1-6020-A}{
		\textit{OEL Display Module Product Specification},
		{September 2006},
		\textbf{Univision Technology Inc}
	}			
\end{thebibliography}

\end{document}
