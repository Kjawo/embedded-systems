\documentclass{classrep}
\usepackage[utf8]{inputenc}
\frenchspacing

\usepackage{graphicx}
\usepackage{subfig}
\usepackage{float}
\usepackage[usenames,dvipsnames]{color}
\usepackage[hidelinks]{hyperref}

\usepackage{adjustbox}

\usepackage{amsmath, amssymb, mathtools}

\usepackage{hyperref}

\usepackage{fancyhdr, lastpage}
\pagestyle{fancyplain}
\fancyhf{}
\renewcommand{\headrulewidth}{0pt}
\cfoot{\thepage\ / \pageref*{LastPage}}


\studycycle{Informatyka, studia dzienne, I st.}
\coursesemester{IV}

\coursename{Systemy wbudowane}
\courseyear{2018/2019}

\courseteacher{dr inż. Michał Morawski}
\coursegroup{środa, 12:15}

\author{%
	\textbf{Grupa D07}\\\\
	\studentinfo[216879@edu.p.lodz.pl]{Przemysław Rudowicz}{216879} - lider\\
	\studentinfo[216782@edu.p.lodz.pl]{Konrad Jaworski}{216782}\\
	\studentinfo[216866@edu.p.lodz.pl]{Jakub Plich}{216866}%
}

\title{Dokumentacja projektu gry Snake\\
		LPC1768/9}
\begin{document}
	\thispagestyle{fancyplain}
	\maketitle
	

	
	\newpage
	\tableofcontents
	\newpage
	
	\section{Podział obowiązków}
	{
	\subsection{Wykorzystane funkcjonalności}
		\begin{center}
			\begin{tabular}{|c|c|}
				\hline 
				\textbf{Funkcjonalność} & \textbf{Osoba za nią odpowiedzialna} \\ 
				\hline 
				GPIO (joystick) & Konrad Jaworski \\ 
				\hline 
				Akcelerometr & Konrad Jaworski \\ 
				\hline 
				Głośnik & Konrad Jaworski \\ 
				\hline 
				Timer & Przemysław Rudowicz \\ 
				\hline 
				OLED & Przemysław Rudowicz \\ 
				\hline 
				SSP/SPI & Przemysław Rudowicz \\ 
				\hline 
				Czujnik światła & Jakub Plich \\ 
				\hline 
				pca9532 & Jakub Plich \\ 
				\hline 
				I\textsuperscript{2}C & Jakub Plich \\ 
				\hline 
			\end{tabular} 	
		\end{center}
	}

	\subsection{Podział obowiązków}
\begin{center}
			\begin{tabular}{|c|c|}
			\hline 
			\textbf{Imię i nazwisko} & \textbf{Procentowy udział w pracy} \\ 
			\hline 
			Konrad Jaworski & 33\% \\ 
			\hline 
			Przemysław Rudowicz & 34\% \\ 
			\hline 
			Jakub Plich & 33\% \\ 
			\hline 
		\end{tabular} 
\end{center}
\newpage

	\section{Opis działania programu}
	
		\subsection{Instrukcja użytkownika}
		
		\subsection{Opis algorytmu}
		
	\section{Funkcjonalności}
		\subsection{GPIO (joystick)}
		GPIO (oznacza general-purpose input/output) - interfejs wejścia/wyjścia ogólnego przeznaczenia. 
		
		\subsection{Głośnik}
	
		\subsection{Akcelerometr}
	
		
		\subsection{Timer}
		
		\subsection{OLED}
		
		\subsection{SSP/SPI}
		
		\subsection{Czujnik światła}
		
		\subsection{PCA9532}
		
		\subsection{I\textsuperscript{2}C}
		
		\subsection{}
	\section{Analiza FMEA}
	

	
	\hspace*{-85pt}\begin{tabular}{|c|c|c|c|c|c|}
		\hline 
		Ryzyko & Prawdopodobieństwo & Znaczenie & (Samo)Wykrywalność & Iloczyn & Reakcja \\ 
		\hline 
		Uszkodzenie joysticka &  &  &  &  &  \\ 
		\hline 
	\end{tabular} 


\end{document}
